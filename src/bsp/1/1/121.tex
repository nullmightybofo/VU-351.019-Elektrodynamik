\begin{question}[section=1,subsection=12,name={Elementare Vektoralgebra},difficulty=3,type=exercise,tags={20110509}]
In Bezug auf eine kartesische Basis sind die Vektoren
$\vec a = \vec e_x + \vec e_y + \vec e_z$, $\vec b = \vec e_x - \vec e_y$, $\vec c = \vec e_x + 2 \vec e_y - 2 \vec e_z$ 
gegeben. Berechnen Sie $ \vec a + \vec b$, $ \vec a + \vec c$, $| \vec a - \vec c |$, $\vec a \cdot \vec b$, $ \vec a \cdot \vec c$, den Kosinus des Winkels zwischen $\vec a$ und $\vec c$, $ \vec a \times \vec b$, den Sinus des Winkels zwischen $\vec a$ und $\vec b$, $\vec a \cdot (\vec b \times \vec c)$, $\vec a \times ( \vec b \times \vec c)$, $(\vec a \times \vec b ) \times \vec c$.
\\ \textbf{Hinweis:}\\
Definition der Verkn\"upfungen ``+'', ``$\cdot$'', ``$\times$''  und des Vektorbetrages. Orthonormalit\"at der kartesischen Basisvektoren, Rechtsschraube.
\end{question}
\begin{solution}
	\begin{align}
	\vec a + \vec b &= 2 \vec e_x + 1 \vec e_z\\
	\vec a \cdot \vec c &= 1 \vec e_x + 2 \vec e_y - 2 \vec e_z\\
	\vec a + \vec c &= 2 \vec e_x + 3 \vec e_y - 1 \vec e_z\\
	|\vec a - \vec c | &= |1 \vec e_y + 3 \vec e_z | = \sqrt{1^2 + 3^2} = \sqrt{10}\\
	\cos(\alpha) &= \frac{\vec a \cdot \vec c}{||\vec a ||_2 ||\vec c||_2}= \frac{1}{\sqrt{3}\sqrt{9}}=\frac{1}{3 \sqrt{3}}\\
	\vec a \cdot \vec b &= 1  - 1 = 0\\
	\vec a \times \vec b &= \vec e_x + \vec e_y - 2 \vec e_z \\
	\vec a \cdot ( \vec b \times \vec c ) &= \left ( \begin{array}{c}1\\1\\1 \end{array} \right ) \cdot \left ( \left ( \begin{array}{c}1\\-1\\0 \end{array} \right ) \times \left ( \begin{array}{c}1\\2\\-2 \end{array} \right ) \right ) = \left ( \begin{array}{c}1\\1\\1 \end{array} \right ) \cdot \left ( \begin{array}{c}2\\2\\3 \end{array} \right ) = 7\\
	\vec a \times ( \vec b \times \vec c) &= \left ( \begin{array}{c}1\\1\\1 \end{array} \right ) \times \left ( \begin{array}{c}2\\2\\3 \end{array} \right ) = \left ( \begin{array}{c}1\\-1\\0 \end{array} \right )\\
	(\vec a \times  \vec b) \times \vec c &= \left ( \begin{array}{c}1\\1\\-2 \end{array} \right ) \times \left ( \begin{array}{c}1\\2\\-2 \end{array} \right ) = \left ( \begin{array}{c}2\\0\\1 \end{array} \right )
\end{align}
\end{solution}