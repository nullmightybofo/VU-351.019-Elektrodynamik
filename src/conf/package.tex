% TU Wien, VU-351.019 Elektrodynamik, Formelsammlung, Prüfungsbeispiele und Lösungen
%Copyright (C) 2016  Painkilla@ www.et-forum.org
%
%This program is free software: you can redistribute it and/or modify
%it under the terms of the GNU General Public License as published by
%the Free Software Foundation, either version 3 of the License, or
%(at your option) any later version.
%
%This program is distributed in the hope that it will be useful,
%but WITHOUT ANY WARRANTY; without even the implied warranty of
%MERCHANTABILITY or FITNESS FOR A PARTICULAR PURPOSE.  See the
%GNU General Public License for more details.
%
%You should have received a copy of the GNU General Public License
%along with this program.  If not, see <http://www.gnu.org/licenses/>.

%----------------------------------------------------------------------------------------
%	PACKAGES AND OTHER DOCUMENT CONFIGURATIONS
%----------------------------------------------------------------------------------------
% Consider using KOMA-Script classes, e.g. scrartcl. May want to check: http://www.komascript.de/release
\documentclass[12pt, a4paper]{article} % Default font size is 12pt, it can be changed here
\usepackage[ngerman]{babel}
\usepackage[T1]{fontenc}
\usepackage[utf8]{inputenc}
\usepackage[pdftex]{graphicx}
\usepackage{pgfplots}
\usepackage{tikz}
\usepackage{verbatim}
\usepgfplotslibrary{polar}
\usetikzlibrary{calc}
\usepackage{amsmath}
\usepackage{amssymb}
\DeclareMathOperator{\arccot}{arccot}
%\DeclareMathOperator{\sinh}{sinh}
%\DeclareMathOperator{\cosh}{cosh}
%\DeclareMathOperator{\tanh}{tanh}
\usepackage{mathrsfs}
\pgfplotsset{compat=1.8}
\usepackage{textcomp}  % what do you need this for?

% geometry package is afaik uneccessary if you only want to set the paper size, put as option in e.g. article
% \usepackage{geometry} % Required to change the page size to A4
% \geometry{a4paper} % Set the page size to be A4 as opposed to the default US Letter

\usepackage{lmodern}  % Latin Modern Roman, imo better fonts than original CMR

% graphicx already loaded above
% \usepackage{graphicx} % Required for including pictures

\usepackage{float} % Allows putting an [H] in \begin{figure} to specify the exact location of the figure

\usepackage{exsheets}
\DeclareQuestionClass{difficulty}{difficulties}
\DeclareQuestionClass{section}{sections}
\DeclareQuestionClass{subsection}{subsections}
\DeclareQuestionClass{type}{types}
\usepackage{tabularx}
\newcolumntype{L}[1]{>{\raggedright\arraybackslash}p{#1}} % linksbündig mit Breitenangabe
\newcolumntype{C}[1]{>{\centering\arraybackslash}p{#1}} % zentriert mit Breitenangabe
\newcolumntype{R}[1]{>{\raggedleft\arraybackslash}p{#1}} % rechtsbündig mit Breitenangabe
\linespread{1.2} % Line spacing
\usepackage{hyperref}
\hypersetup{pdftitle={VU 351.019 Elektrodynamik}, pdfauthor={Painkilla@www.et-forum.org}, pdfkeywords={351.019,Elektrodynamik}, pdfcreator={pdflatex}, pdfborderstyle={/S/U/W 1}}

%\graphicspath{{Pictures/}} % Specifies the directory where pictures are stored
\numberwithin{equation}{subsection}

\makeatother